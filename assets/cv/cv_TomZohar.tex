%%%%%%%%%%%%%%%%%%%%%%%%%%%%%%%%%%%%%%%%%
% Medium Length Graduate Curriculum Vitae
% LaTeX Template
% Version 1.1 (9/12/12)
%
% This template has been downloaded from:
% http://www.LaTeXTemplates.com
%
% Original author:
% Rensselaer Polytechnic Institute (http://www.rpi.edu/dept/arc/training/latex/resumes/)
%
% Important note:
% This template requires the res.cls file to be in the same directory as the
% .tex file. The res.cls file provides the resume style used for structuring the
% document.
%
%%%%%%%%%%%%%%%%%%%%%%%%%%%%%%%%%%%%%%%%%

%----------------------------------------------------------------------------------------
%	PACKAGES AND OTHER DOCUMENT CONFIGURATIONS
%----------------------------------------------------------------------------------------

\documentclass[margin]{res} % Use the res.cls style, the font size can be changed to 11pt or 12pt here
%\documentclass[margin]{article}

%\usepackage{helvet} % Default font is the helvetica postscript font
\usepackage{newcent} % To change the default font to the new century schoolbook postscript font uncomment this line and comment the one above
\usepackage{comment}
\usepackage{etoolbox}
 \usepackage{multicol}
\usepackage{hyperref}
\hypersetup{%
  colorlinks=true,% hyperlinks will be black
  linkcolor=blue,% hyperlink text will be green
  %linkbordercolor=red,% hyperlink border will be red
  %pdfborderstyle={/S/U/W 1}% border style will be underline of width 1pt
}

\AfterEndEnvironment{resume}{\vspace{-5\baselineskip}}

\setlength{\textwidth}{5in} % Text width of the document

\newcommand{\rootFolder}{/Users/tomzohar/Dropbox/}

\begin{document}

%----------------------------------------------------------------------------------------
%	NAME AND ADDRESS SECTION
%----------------------------------------------------------------------------------------

\moveleft.5\hoffset\centerline{\large\bf Tom Zohar} % Your name at the top
 
\moveleft\hoffset\vbox{\hrule width\resumewidth height 1pt}\smallskip % Horizontal line after name; adjust line thickness by changing the '1pt'
\moveleft.5\hoffset\centerline{CEMFI} % Your address
\moveleft.5\hoffset\centerline{C. Casado del Alisal, 5, Madrid, Spain, 28014}
\moveleft.5\hoffset\centerline{+34-914-290551}
\moveleft.5\hoffset\centerline{\textcolor{blue}{tom.zohar@cemfi.es}}
\moveleft.5\hoffset\centerline{\href{https://tomzohar.com}{website}}


%----------------------------------------------------------------------------------------

\begin{resume}

%----------------------------------------------------------------------------------------
%	EDUCATION SECTION
%----------------------------------------------------------------------------------------

\section{Employment}

Assistant Professor of Economics, CEMFI \hfill September 2021

%----------------------------------------------------------------------------------------
%	EDUCATION SECTION
%----------------------------------------------------------------------------------------

\section{EDUCATION}

PhD in Economics, Stanford University \hfill June 2021

Visiting Student in Economics, UC Berkeley  \hfill Fall 2012 

BA in Economics, Interdisciplinary Center (IDC) - Summa Cum Laude \hfill 2013

%----------------------------------------------------------------------------------------
%	DISSERTATION COMMITTEE
%----------------------------------------------------------------------------------------
\begin{comment}
\section{DISSERTATION COMMITTEE}


\begin{multicols}{2}
	\begin{itemize}
		 \item[] Prof. Ran Abramitzky (Primary) \\
		Economics Department, Stanford University \\
		(650) 723-9276 \\
		\textcolor{blue}{ranabra@stanford.edu} \\

		 \item[] Asst. Prof. Petra Persson \\
		Economics Department, Stanford University \\
		(650) 723-4116 \\
		\textcolor{blue}{perssonp@stanford.edu} \\

		 \item[] Prof. Liran Einav\\
		Economics Department, Stanford University \\
		(650) 723-3704 \\
		\textcolor{blue}{leinav@stanford.edu} \\

		 \item[] Asst. Prof. Isaac Sorkin\\
		Economics Department, Stanford University \\
		(608) 440-0052 \\
		\textcolor{blue}{sorkin@stanford.edu} \\

	\end{itemize}
\end{multicols}
\end{comment}

%----------------------------------------------------------------------------------------
%	OBJECTIVE SECTION
%----------------------------------------------------------------------------------------
 
\section{FIELDS}  
 Labor and Public Economics: Focus on Labor-Market Inequality, Economics of the Household, and Reproductive Health
 
%----------------------------------------------------------------------------------------
% Working Papers
%----------------------------------------------------------------------------------------

\section{ RESEARCH} 

\subsection{WORKING PAPERS}

\textit{Out of Labor and into the Labor Force? The Role of Abortion Access, Social Stigma, and Financial Constraints} (with Nina Brooks) 
%\href{https://web.stanford.edu/~tzohar/Tom%20Zohar%20-%20JMP.pdf}{Latest Version}

%\input{\rootFolder/Research/abortionDemand/Writing/manuscript/sections/abstract}

%The Unplanned Child Penalty (with Nina Brooks)

\subsection{WORK IN PROGRESS}

\textit{Decomposing the Intergenerational Transmission of Income} (with Caue Dobbin) 
%\href{https://www.dropbox.com/s/xodwuv0ui28xt8t/IGM_decomposition.pdf?dl=0}{Latest Version}

%\input{\rootFolder/Research/reassIneq/intergenMob/texArticle/IGM_decomposition/sections/abstract}

%\input{\rootFolder/Research/headOfTheFox/Writing/manuscript/sections/abstract}

\textit{Why is There Positive Assortative Matching?} (with Caue Dobbin) 
%\href{https://www.dropbox.com/s/a49d8sf88gda0uu/2020_02_24_devTea.pdf?dl=0}{Slides}

\textit{Head to the Foxes or Tail to the Lions? The Importance of within Location Ordinal Rank in Childhood Environment} (with Tslil Aloni and Hadar Avivi)


%It has been extensively documented that a substantial share of income inequality can be associated with the sorting of workers to firms. However, not all dimensions of firm heterogeneity are captured in the data. Amenities, such as bonuses, health care benefits and in-kind payments, are rarely observed; while non-monetary amenities, such as prestige, never are. Therefore, welfare inequality can be distinct from income inequality if high earners are sorted into high or low amenities firms. We investigate the issue by comparing the flows of different types of workers between firms. We find that poor workers who are employed in low-pay firms have large earnings gains when they change firms and, as a consequence, are very likely to leave such firms. Rich workers, on the other hand, are not more likely to leave low-pay firms than high-pay ones, even though they also enjoy large earnings gains when they leave low-pay firms. From a revealed preference perspective, we conclude that top- earners receive a large share of their compensation as non-observed amenities. As a consequence, inequality measures based on earnings will be downward biased.


%----------------------------------------------------------------------------------------
% 	TEACHING
%----------------------------------------------------------------------------------------

\section{TEACHING EXPERIENCE}

Labor Economics (Graduate) \hfill 2021

 Head-TA of Stanford's Economics Department \hfill 2018 - 2020

 Labor Economics (Undergrad), Stanford, Prof. John Pencavel \hfill Fall 2017

%----------------------------------------------------------------------------------------
%	RELEVANT EXPERIENCE SECTION
%----------------------------------------------------------------------------------------
 
\section{OTHER RESEARCH AND PROFESSIONAL EXPERIENCE}
 Research Assistant, Prof. Ran Abramitzky (Stanford) \hfill 2016-2018

 Field Research Assistant (Tanzania), Prof. Melanie Morten \hfill Summer 2016
 
 Management Consultant, Deloitte \hfill 2014-2015 

 Lab Research Assistant, Prof. Tali Regev (IDC) \hfill 2012-2013

%----------------------------------------------------------------------------------------
%	FELLOWSHIPS AND AWARDS
%----------------------------------------------------------------------------------------

\section{FELLOWSHIPS, HONORS AND AWARDS}

Out of Labor and into the Labor Force: Abortion Access \& The Unplanned Child Penalty
Donor: 
\begin{itemize}
	\item Leonard W. Ely and Shirley R. Ely Graduate Student Fellowship (\$29,958)
	\item Shultz Fellowship - November 2018 (\$12,500)
    \item Shultz Fellowship - April 2018 (\$8,500)
    \item Graduate Research Opportunity Funds (\$4,500)
    \item FSI GOES large research grant (\$5,700)
\end{itemize}	

Reassesing inequality: A revealed preference approach
Donor: 
\begin{itemize}
	\item Shultz Fellowship - April 2018 (\$18,516)
	\item Shultz Fellowship - Nov 2017 (\$46,840)
	\item Stanford Center for Computational Social Science (\$2,000) 
\end{itemize}	

Honors Fellowship, IDC (\$13,500) 

Berkeley Economics Semester Abroad Program Fellowship (\$2,000) 

Rotary Fellowship (\$1,200) 

%----------------------------------------------------------------------------------------
%	PRESENTATION SECTION
%----------------------------------------------------------------------------------------
 
\section{PRESENTATIONS} 
\vspace{5mm} 
Out of Labor and into the Labor Force: Abortion Access \& The Unplanned Child Penalty
\begin{itemize}
	\item ASSA, Jan 2022, Virtual (forthcoming)
	\item NBER SI, July 2021, Virtual
	\item CSWEP, June 2021, Virtual
	\item VERB, May 2021, Virtual
	\item Bank of Israel, April 2021, Virtual
	\item APPAM Fall Research Conference, November 2020, Virtual
	\item PAA Annual Meeting, April 2020, Virtual
	\item APPAM International Confererence, July 2019, University Pompeu Fabra
	\item IRES Graduate Workshop, May 2019, Chapman University
	\item APPAM CA Student Confererence, April 2019, UC Irvine
\end{itemize}

 
%----------------------------------------------------------------------------------------
%	Other skills
%----------------------------------------------------------------------------------------
\section{OTHER} 
Nationalities: Israeli, Portuguese (Pending) \\
Languages: Hebrew (native), English (fluent) \\
Software: R, Stata, Matlab, Python, \LaTeX, Git

%----------------------------------------------------------------------------------------
%	Last Update
%----------------------------------------------------------------------------------------
\section{LAST UPDATE} 
October 2021
%----------------------------------------------------------------------------------------

\end{resume}
\end{document}